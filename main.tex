%
% File acl2021.tex
%
%% Based on the style files for EMNLP 2020, which were
%% Based on the style files for ACL 2020, which were
%% Based on the style files for ACL 2018, NAACL 2018/19, which were
%% Based on the style files for ACL-2015, with some improvements
%%  taken from the NAACL-2016 style
%% Based on the style files for ACL-2014, which were, in turn,
%% based on ACL-2013, ACL-2012, ACL-2011, ACL-2010, ACL-IJCNLP-2009,
%% EACL-2009, IJCNLP-2008...
%% Based on the style files for EACL 2006 by 
%%e.agirre@ehu.es or Sergi.Balari@uab.es
%% and that of ACL 08 by Joakim Nivre and Noah Smith

\documentclass[11pt,a4paper]{article}
\usepackage[hyperref]{acl2021}
\usepackage{times}
\usepackage{latexsym}
\renewcommand{\UrlFont}{\ttfamily\small}
\usepackage{comment}
\usepackage{subfig}
\usepackage{setspace}
\usepackage{graphicx}
\usepackage{xcolor}
\usepackage{mathtools}

\usepackage{multirow}

\newcommand{\han}[1]{\textcolor{red}{#1}}

% This is not strictly necessary, and may be commented out,
% but it will improve the layout of the manuscript,
% and will typically save some space.
\usepackage{microtype}

%\aclfinalcopy % Uncomment this line for the final submission
%\def\aclpaperid{***} %  Enter the acl Paper ID here

%\setlength\titlebox{5cm}
% You can expand the titlebox if you need extra space
% to show all the authors. Please do not make the titlebox
% smaller than 5cm (the original size); we will check this
% in the camera-ready version and ask you to change it back.

\newcommand\BibTeX{B\textsc{ib}\TeX}

\title{Enhancing Spoken Language Fluency Assessment via Task Fusion}

\author{First Author \\
  Affiliation / Address line 1 \\
  Affiliation / Address line 2 \\
  Affiliation / Address line 3 \\
  \texttt{email@domain} \\\And
  Second Author \\
  Affiliation / Address line 1 \\
  Affiliation / Address line 2 \\
  Affiliation / Address line 3 \\
  \texttt{email@domain} \\}

\date{}

\begin{document}
\maketitle
\begin{abstract}
Assisting in foreign language learning is one of
the major areas in which natural language processing technology can contribute. For spoken language fluency assessment tasks, state-of-the-art deep learning models are trained based on large amount of data, which requires heavy human annotation work. However, there might exist inherent labeling bias from human annotators. In this work, we propose a novel framework which combines the tasks to jointly predict the labels which are annotated on both audio inputs and text inputs. Moreover, by jointly model the audio and text input data, we further improve the performance of language fluency assessments. From experiments, we validate the effectiveness of our proposed method.


\end{abstract}




\section{Introduction}


\han{Only Fluency or Can be Proficiency? Verbal fluency might not be proper, it appears more in medical studies}

Spoken language fluency assessment~\cite{aldhanhani2020theories} is an essential task for educational AI, which helps language learners to improve their learning efficiency and experience. Until now, deep learning based assessment systems~\cite{metallinou2014using, cheng2015deep} were developed to achieve the state of the art performance. Typically, the assessment models are trained based on a large amount audio speech records with each record is labeled by human annotators to evaluate the speech's fluency. 


However, because of the large amount of training data to be labeled and the spoken language fluency is always subjective to the individual judgment of human annotators, there usually exists severe biases in the labeling process. For example, imagine that different annotators are assigned to score different divisions of speeches, their judgements can be fundamentally influenced by comparisons inside the assigned division, but
not reflect the overall fluency level in the whole population. In this way, the performance of fluency assessment models can be degraded.



In this work, we propose a novel framework to strengthen the ``objectivity'' of fluency labeling process. We aim to introduce auxiliary fluency scores which are annotated on the ASR outputted texts for the existing speeches. Intuitively, people's judgement on text fluency might have inherent different criterion with their judgements on speaking fluency. For example, \han{Some intuitive convincing reasons}. Thus, by including human's evaluation from diverse perspectives, we could effectively strengthen the ``objectivity'' in annotation. 

In this work we design a novel framework which can ....inspired by ESMM.


Moreover, we further improve performance of our fluency assessment models by jointly model the audio inputs and text inputs together. 



\newpage


\section{The Proposed Approach}
In this section, we describe the general framework for jointly optimizing the primary network

\subsection{Notations}


\subsection{Auxiliary Learning Framework}


\subsection{Joint Training}


\section{Experiments}
In this section, we evaluate the Auxi-Merge Learning framework in a series of model structures for spoken language fluency assessment.

\subsection{Experimental Setup}
\textbf{Datasets.} During our survey, no public datasets with overall fluency labels and text fluency labels based on ASR text simultaneously are found in spoken language fluency assessment area.
To evaluate the proposed framework, we collected a real-world dataset from a K-12 education platform. 
The dataset consists of many audios and corresponding ASR text of english oral homeworks from students of grade 1 to grade 3. 
Every student is asked to speak some sentences based on a given topic and a template. 
In order to obtain as objective fluency labels as possible, the datasets are divided into many pieces, and part of the pieces are assigned to several annotators randomly. 
Finally, we choose the data which is consistent among five or more annotators as our training or evaluating dataset. 

Each item of the dataset consists of four parts: a overall fluency label, a text fluency label, a audio file in MP3 format, and a paragraph of ASR text generated from the audio.

\textbf{Baselines.} In order to verify the universality of the proposed framework, we construct many heterogeneous models for fluency assessment as competitors. Among these baselines, ~\cite{didi} is one of them, which can be regarded as the state of the art method of multi-model fusion, and the construction of other models refers to the design of some components in ~\cite{didi}. These models can generally be divided into two types: one is based on some traditional features extraction methods, and the other one is End-to-End  structure based on some pre-train models. And in each type, models can be further classified into single-model and multi-model according to the included modality: text or/and audio. All models of baselines and their types are listed in Table~\ref{experimental-baselines}. 

(1)T-Rand: The model's inputs are only ASR text generated from audios. For word representation, we use a 300-dimensional random embedding as the text embedding. And then the text embeddings are taken into a BiLSTM with 100 hidden units. Finally, the outputs of the BiLSTM are taken max-pooling and the results of binary classification are generated based on a fully-connected layer with a $200\times2$ weight matrix corresponding to the number of hidden states and the the number of classes. To train the model, we use Adam optimization with the learning rate of 0.0005.

(2)A-PAA: The model is only based on audios. Refer to the model in ~\cite{didi}, we remove parts of text encoding and multi-head attention from it, just retain the audio features extraction, encoding based on BiLSTM and fully-connected layer for binary classification.

(3)TA-DiDi: The model is a implementation of ~\cite{didi}.

(4)T-Bert: The model is a counterpart to the T-Rand. The difference between them lies in the way of text embedding. T-Bert utilizes the Bert pre-training model to generage the text embedding instead of random embedding.

(5)A-VGGish: The model is a counterpart to the A-PAA, so the structure is also similar. But for audio features generation, it does not use the methods of extracting traditional acoustic features, but directly generates feature embeddings of audio windows based on VGGish audio pre-training model.

(6)TA-BV
 based on overall fluency labels or/and text fluency labels

\textbf{Metric.}
Since the proposed framework is still in the field of auxiliary learning, that is to say, the ultimate goal of the task is only one, which is to predict whether the overall fluency.

 \begin{table}
\centering
\begin{tabular}{ccc}
\hline \textbf{Model} & \textbf{Pre-trainning} & \textbf{Modality} \\ \hline
T-Rand & No & Text \\
A-PAA & No & Audio \\
TA-DiDi & No & Text\&Audio \\
\hline
T-Bert  & Yes & Text \\
A-VGGish & Yes & Audio \\
TA-BV & Yes & Text\&Audio \\
\hline
\end{tabular}
\caption{\label{experimental-baselines} Baselines. }
\end{table}

\subsection{Results}
Table~\ref{experimental-results} shows results of all above methods on the dataset. (1) Overall effect. (2) pretrain Detail (3) modality detail

\begin{table*}
\centering
\begin{tabular}{cccccc}
\hline
\textbf{Model} & \textbf{Exp} & \textbf{AUC} & \textbf{Precision} & \textbf{Recall} & \textbf{F1} \\
\hline
\multirow{2}{*}{T-Rand} & Baseline & 0.7587 & 0.828 & 0.8524 & 0.84 \\
& Auxi-Merge & 0.7822 & 0.8357 & 0.8635 & 0.8494 \\
\hline
\multirow{2}{*}{A-PAA} & Baseline & 0.9025 & 0.9101 & 0.9336 & 0.9217 \\
& Auxi-Merge & & & & \\
\hline
\multirow{2}{*}{TA-DiDi} & Baseline & 0.8599 & 0.9292 & 0.7749 & 0.8451 \\
& Auxi-Merge & 0.881 & 0.924 & 0.8708 & 0.8868 \\
\hline
\multirow{2}{*}{T-Bert} & Baseline & 0.8231 & 0.8833 & 0.8376 & 0.8598 \\
& Auxi-Merge & & & & \\
\hline
\multirow{2}{*}{A-VGGish} & Baseline & 0.9407 & 0.9384 & 0.9557 & 0.9470 \\
& Auxi-Merge & & & & \\
\hline
\multirow{2}{*}{TA-BV} & Baseline & & & & \\
& Auxi-Merge & & & & \\
\hline
\end{tabular}
\caption{\label{experimental-results} Comparision of different models}
\end{table*}

\section{Conclusions and Future Work}

\newpage
\bibliographystyle{acl_natbib}
\bibliography{our_ref}

%\appendix



\end{document}
